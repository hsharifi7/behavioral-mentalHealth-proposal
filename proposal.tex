%%%%%%%%%%%%%%%%%%%%%%%%%%%%%%%%%%%%%%%%%%%%%%%%%%%%%%%%%%%%%%%%%%%%%%%%%%%%%%%%
%2345678901234567890123456789012345678901234567890123456789012345678901234567890
%        1         2         3         4         5         6         7         8

\documentclass[letterpaper, 10 pt, conference]{ieeeconf}  % Comment this line out
                                                          % if you need a4paper
%\documentclass[a4paper, 10pt, conference]{ieeeconf}      % Use this line for a4
                                                          % paper

\IEEEoverridecommandlockouts                              % This command is only
                                                          % needed if you want to
                                                          % use the \thanks command
\overrideIEEEmargins
% See the \addtolength command later in the file to balance the column lengths
% on the last page of the document

 
% The following packages can be found on http:\\www.ctan.org
%\usepackage{graphics} % for pdf, bitmapped graphics files
%\usepackage{epsfig} % for postscript graphics files
%\usepackage{mathptmx} % assumes new font selection scheme installed
%\usepackage{times} % assumes new font selection scheme installed
%\usepackage{amsmath} % assumes amsmath package installed
%\usepackage{amssymb}  % assumes amsmath package installed
\usepackage[final]{pdfpages}
\usepackage{caption, rotating}
\usepackage{graphics}
\usepackage{array}
\usepackage[export]{adjustbox}
\usepackage[T1]{fontenc}
 
\title{\LARGE \bf
Lowering Depression and Anxiety: A Quantititave Research on the Relationship of Seven Common Habits 
on Human's Mental Health
}
 
\author{Dang Quang Hoang, Karthikeyan Marikrishnan \\ Yuqing Ren, Muhammad Hamza Raza, Hadi Sharifi}



\def\@testdef #1#2#3{%
  \def\reserved@a{#3}\expandafter \ifx \csname #1@#2\endcsname
 \reserved@a  \else
\typeout{^^Jlabel #2 changed:^^J%
\meaning\reserved@a^^J%
\expandafter\meaning\csname #1@#2\endcsname^^J}%
\@tempswatrue \fi}

\begin{document}



\maketitle
\thispagestyle{empty}
\pagestyle{empty}


%%%%%%%%%%%%%%%%%%%%%%%%%%%%%%%%%%%%%%%%%%%%%%%%%%%%%%%%%%%%%%%%%%%%%%%%%%%%%%%%
% \begin{abstract}

% This electronic document is a ÒliveÓ template. The various components of your paper [title, text, heads, etc.] are already defined on the style sheet, as illustrated by the portions given in this document.

% \end{abstract}


%%%%%%%%%%%%%%%%%%%%%%%%%%%%%%%%%%%%%%%%%%%%%%%%%%%%%%%%%%%%%%%%%%%%%%%%%%%%%%%%
\section{INTRODUCTION AND PROBLEM STATEMENT}
Depression and anxiety are two widespread types of disorders that endures a tremendous consequences on human life.  
The World Health Organization (WHO) has ranked depression as the fourth leading cause of human disability.
By 2020, it reaches to second leading cause \cite{kessler2013epidemiology}. Many researches touch the symptoms of anxiety and depression.
As an example, depression causes health 
complications \cite{verma2017impact}, cardiovascular diseases \cite{bradley2015depression}, in some cases increases 
the risk of cardiovascular by 80\% \cite{penninx2017depression}. In case of anxiety, in average, up to 33.7\% of 
the human populations expereinces it in their life time \cite{bandelow2015epidemiology}. Anxiety no only affects physically
but causes learning and reasoning incapacities \cite{spielberger2013effects}\cite{darke1988effects}. 
Clearly they are two big risks factors for human life.
This proposal analyzes data from the Behavioral Risk Factor Surveillance System (BRFSS) of several years. 
It tries to find a relationship between six habit factors (physical activity, eating disorder, 
smoking, drinking alcohol, social media, and education/technology) and depression and anxiety. It proposes a
solutions that could lead to reduction of depression and anxiety in the society. 

\section{OBJECTIVE}

\noindent\textit{$\circ$ What this research is trying to accomplish?} \newline
\textnormal{
Identifying the relation between the six factors and depression and anxiety. 
And provide guidelines based on the six factors to reduce depression and anxiety in human life.
}

\setlength{\parskip}{1em} %give space between paragraph. Except for the first one above.

\par\noindent\textit{$\circ$ How is research in this field is done today; what are the limits of current practice?}\newline
\textnormal{
Majority of research papers on anxiety and depression covers few variables. 
This limits the scope of influence in exacerbating these disorder. 
}
\par\noindent\textit{$\circ$ What's new to this research? Why will it be successful?}\newline
\textnormal{
This research investigates more recent dominant habits. The outcome of the 
research provides guidance for larger body of human society. The key to success 
of this research is data and linking data to the right conclusion. 
BRFSS is a known data that would pave the path for success of the research. 
}

\par\noindent\textit{$\circ$ Who cares?}\newline
\textnormal{
The general public, medical society, insurance industry, and corporation. Depression 
and anxiety are felt in each and every part of the human life and it is in interest 
of all above mentioned to control or reduce outcome of anxiety and depression affect.
}
\par\noindent\textit{$\circ$ If this research is successful, what difference and impact will it make, and how do you measure them?}\newline
\textnormal{
Different sectors of 
human society can use the guidance to avoid anxiety and depression and identify them at the early 
stages of the disease. It will provide recipes to various human resource organization 
on how to avoid anxiety and depression. Surveys such as BRFSS and local and internal 
surveys can provide a great measure on how this research impacted them.
}
\par\noindent\textit{$\circ$ What are the risks and payoffs?}\newline
\textnormal{
The risk is to convince mass public, human resource organizations, and small 
to large companies that the results of this research will indeed assist them 
get better and faster results. The payoffs are happier work, happier life, 
happier families, and happier society.  
}
\par\noindent\textit{$\circ$ How much will it cost?}\newline
\textnormal{
The biggest cost is the time. The data is available, but it 
needs to be cleaned, information to be extracted and analyzed. 
At this stage, we anticipates 150 to 200 hours of scientific work. 
}
\par\noindent\textit{$\circ$ How long will it take?}\newline
\textnormal{
The research can be done in 3 to 6 months. But we are going to start
with only 6 factors and hopefully start the spark for future research. 
}
\par\noindent\textit{$\circ$ How will progress be measured.}\newline
\textnormal{
The progress of this research is measured by first establishing a clear connection 
between the six habits and anxiety and depression. Second understanding how these 
factors can decrease desease. And third provides the golden 
guidelines for various parties.
}

\setlength{\parskip}{.5em} %stop giving space
\section{LITERATUR REVIEW}

\par\noindent\textit{$\circ$ The effects of physical activity?}\newline
We have studied three research papers.  
The \cite{strohle2009physical} paper provided a survey on the association 
between physical and therapeutic activity on depression and anxiety. 
The \cite{mammen2013physical} paper analyzes multiple databases to identify factors causing depression as 
well as examine whether physical activity prevents depression. Both show that
physical activity reduces and, in some cases, prevents depression and anxiety. 
The criticism on these papers are that they do not pay adequate attention to symptoms 
and approaches to deal with depression and anxiety as well as benefits of exercise training.
Interestingly, the \cite{van2013exploratory} found that there is no 
relation between vigorous physical activity and mental health or well-being. We believe 
the reason of this results is the vigorous nature of physical activity. 

\setlength{\parskip}{1em} %give space again

\par\noindent\textit{$\circ$ The effects of alcohol abuse and smoking?}\newline
We picked four papers \cite{jia2018associations}\cite{strine2008depression}\cite{allan2015effects}\cite{patton1996smoking}
, all corroborated our hypothesis that abusing alcohol and smoking leads to anxiety and depression. Two of the 
researches used the BRFSS data set. These are valuable research to us. Almost all of them did show a shortcoming that 
the effects on mental health goes beyond one to two variables. Interestingly, research \cite{patton1996smoking}  
from 96 advised school to look into using smoke to help teenagers cope with depression. We are not going to use this paper. 
Smoking may temporarly alevite depression but it leads to more mental and health symptoms. 


\par\noindent\textit{$\circ$ The effects of socila media and knoledge?}\newline
We have studied three research papers in this topic. They show a strong correlation between 
social media and depression and anxiety. The paper \cite{lin2016association} emphasises on the correlation 
between social media and depression while considering other environmental and other factors such as family and financial. 
The second paper \cite{jelenchick2013facebook} analyzes social networking sites and the relation to depression in older 
adolescents. The participants used have small age difference which lowers the risk of many 
environmental factors skewing the results. The third paper \cite{woods2016sleepyteens} analyzes the use of social media 
and how it relates to depression, anxiety, sleep quality and self-esteem in adolescents. The reseach does 
lack analyzing the effects on day time of the users 

\par\noindent\textit{$\circ$ The effects of technology?}\newline
We have studied three papers \cite{demirci2015relationship}\cite{bjelland2008does}\cite{mezuk2008influence}. All show positive correlation between 
factors such as high usage of smartphone, low education level and type 2 
diabetes, and depression and anxiety. They have confirmed our hypothesis 
that smartphone/education/diabetes are among leading factors of depression 
disorder and anxiety. All three papers touch particular aspects of technology and we think we should follow the same trend.
We may focus on a particular technology, such as cellphone, instead of "technology" in general. 

\par\noindent\textit{$\circ$ The effects of eating disorder?}\newline
The 
first of the three papers \cite{sassaroli2005role} shows that eating disorder leads to stress and anxiety in 
high school girls. The second paper \cite{martz1995relationship} shows that women with eating disorder 
get highly stressed and the stress led to anxiety behaviors. They also concluded that 
traditional female role causes these symptoms. The third paper \cite{striegel2007risk} shows genetically 
some patients are showing symptoms of eating disorder. This genetic issue leads 
to other issues such as depression and anxiety. The criticism we have on these papers that they only pick 
female population. For our research we will use 
these papers nevertheless, we will make sure to use data for both male and female.  

\section{METHODLOGY}
The research requires analyzing large data that are not clean and not organized. 
The main task of our team is to
extract data from the assigned topics, clean the data, put it a format that all 
other teammates will agree upon. From that point, we analyze the data, get conclusions, and produce the guidlines.  
Figure \ref{fig:schedule} shows the details on how various tasks are distributed among team members 
and how the timeline is formed to reach all the deadlines. 

%\includepdf{schedule.pdf}
\captionsetup[figure]{labelformat=empty}
\clearpage 
\begin{figure}[hbt!]
        \centering
        \includepdf[pages=-]{schedule.pdf}
        \addvspace{250pt}
        
        \hspace{-10cm}
        \rotatebox{90}{ Figure ~\ref{fig:schedule}: The schedule of the team for the research.}
        
        %\caption{The Schedule of all tasks defined for the team.}
        \caption{}
        \label{fig:schedule}
\end{figure}

\clearpage 


\bibliography{biblio}
\bibliographystyle{plain}

\end{document}
